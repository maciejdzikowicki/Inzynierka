%%
\chapter{Przegląd literatury}

w latach 60--tych XX~w. Departament Obrony Stanów Zjednoczonych rozpoczął tworzenie sieci komputerowej (ARPANET), której celem było połączenie uniwersytetów oraz innych jednostek realizujących projekty dla armii w~celu umożliwienia im wzajemnego dostępu do mocy obliczeniowej komputerów. Gdy w~późniejszym okresie opracowano rodzinę protokołów TCP/IP i~zaimplementowano je w~systemach UNIX’owych okazało się, że niemal natychmiast sieć ARPANET powiększyła się o~wszystkie lokalne sieci komputerowe zainstalowane w~uniwersytetach stanowych \cite{ss3,ss4,ss5}.

\lipsum[1-1]

\begin{quote}
\scriptsize{
\underline{\textcolor{blue}{Objaśnienie:}}\\[1mm]
\textcolor{blue}{\textbf{Przegląd literatury} (cz. teoretyczna objętość 10--15 str.)}\\[1mm]
\textcolor{blue}{
Rozdział podzielony na 2-4 podrozdziałów musi zawierać odpowiedzi na następujące pytania:
\begin{itemize}
\item  Jaka problematyka z~zakresu informatyki będzie analizowana?
\item  Jakie definicje, pojęcia będą wykorzystane w~pracy? (muszą być wyjaśnione słowa kluczowe zawarte w~tytule pracy)?
\item  Jak problem ewaluował historycznie? Co się zmieniało? Dlaczego? Co było tego przyczyną?
\item  Jacy autorzy (polscy, zagraniczni) zajmowali się w~przeszłości tą problematyką ?
\item  Jakie są poglądy poszczególnych autorów prac naukowych w~poruszanych kwestiach?
\item  Jakie teorie, „\emph{szkoły}" ~aktualnie obowiązują? Jakie ośrodki są wiodące?
\item  Jakie są podobieństwa i~różnice między autorami w~rozpatrywanej problematyce?
\item  Jakie jest stanowisko dyplomanta? Jakie przyjmuje definicje, twierdzenia? Mogą być przyjęte od konkretnych autorów, byleby to zaznaczyć w~tekście (w przypisach).
\end{itemize} }}
\end{quote}

